\documentclass[12pt,a4paper,oneside]{article}
\usepackage[utf8]{inputenc}
\usepackage{amsmath}
\usepackage{amsfonts}
\usepackage{amssymb}
\usepackage{paralist}
\usepackage{graphicx}
\usepackage{hyperref}
\usepackage{booktabs}
\usepackage[top=1cm,left=2cm,right=2cm,bottom=2cm]{geometry}
\usepackage[usenames]{color}
\usepackage{amsthm}

\newcommand{\var}{\text{var}}
\newcommand{\bb}{\boldsymbol}
\renewcommand{\baselinestretch}{1}


\usepackage{txfonts}
\begin{document}

\noindent \textbf{Tarea 2 parte II} \hfill\textbf{Modelos Lineales Generalizados} \\[-2ex]
\rule{\textwidth}{0.5pt}\\

	
\noindent 	\textbf{Profesor}: Graciela Martínez Sánchez, graciela.mtz@ciencias.unam.mx\\
\textbf{Ayudante}: José de Jesús Ojeda Gónzalez, jesusojeda@ciencias.unam.mx\\
\textbf{Alumno}: Cristobal Bautista Hernández.\\
\textbf{Fecha de entrega}: jueves 30 de abril de 2020.\\[2ex]
\noindent \textbf{Instrucciones:} La tarea se entrega de manera individual, ordenada y bien escrita.  La solución de la tarea debe enviarse en un archivo \texttt{.tex}\\

%\centerline{Distribuciones muestrales}

%\noindent \textbf{1. } Muestre que la tasa de cambio $\partial \pi_i/\partial \beta_j$ para el modelo logístico se máximiza en $\pi_i =0.5$.\\

\noindent \textbf{1. } Considere que se ajustó un modelo multinomial logístico  para predecir las preferencias para presidente (demócrata, repúblicanos e independientes), en función del ingreso anual $x$ (en dólares):

\[\begin{array}{cc}
\log \left(\dfrac{\hat \pi_{D}}{\hat \pi_{I}}\right)  = 3.3 - 0.2 x,& \qquad
\log \left(\dfrac{\hat \pi_{R}}{\hat \pi_{I}}\right) = 1.0 + 0.3 x.
\end{array}\]

\begin{compactenum}

\item[(i)] Encuentre una expresión para $\log \left(\dfrac{\hat \pi_D}{\hat \pi_R}\right)$. Interprete la pendiente en la ecuación.
\item[(ii)] Encuentre el rango de $x$ para el cual $\hat \pi_D>\hat \pi_R$.
\item[(iii)]Obtenga una expresión para la probabilidad $\hat \pi_I$.\\
\end{compactenum}

\begin{compactenum}
\item[(i)] 
\begin{proof}[\textcolor{blue}{Solución:}] Observe que $\log\left(\frac{\pi_i}{\pi_j}\right) = \log(\left(\frac{\pi_i}{\pi_k}\right)\left(\frac{\pi_k}{\pi_j}\right)) = log\left(\frac{\pi_i}{\pi_k}\right) + log\left(\frac{\pi_k}{\pi_j}\right) = log\left(\frac{\pi_i}{\pi_k}\right) - log\left(\frac{\pi_j}{\pi_k}\right)$ para algún $k = 1, 2, \dots , J$

De esta manera, aplicando a nuestro problema:

$$\log\left(\frac{\widehat{\pi}_D}{\widehat{\pi}_R}\right) = \log\left(\frac{\widehat{\pi}_D}{\widehat{\pi}_I}\right) - \log\left(\frac{\widehat{\pi}_R}{\widehat{\pi}_I}\right) = 3.3 - 0.2 x - (1.0 + 0.3 x) = 2.3 - 0.5 x \Rightarrow \log\left(\frac{\widehat{\pi}_D}{\widehat{\pi}_R}\right) = 2.3 - 0.5 x$$

De está ecuación se puede decir que entre mayor sea el ingreso anual dolares para una persona, está tendrá con mayor probabilidad a preferir a un presidente republicano que a un demócrata.

\end{proof}

\item[(ii)]
\begin{proof}[\textcolor{blue}{Solución:}] Utilizando la hipótesis y el hecho de que la función $log$ es monótona creciente.
$$ \widehat{\pi}_D > \widehat{\pi}_R \Leftrightarrow \widehat{\pi}_D > \widehat{\pi}_R \left( \frac{\widehat{\pi}_I}{\widehat{\pi}_I} \right) \Leftrightarrow \log \left( \widehat{\pi}_D \right) > \log \left( \widehat{\pi}_I \left( \frac{\widehat{\pi}_R}{\widehat{\pi}_I} \right) \right) = log \left( \widehat{\pi}_I \right) + log \left( \frac{\widehat{\pi}_R}{\widehat{\pi}_I} \right) \Leftrightarrow \log \left( \widehat{\pi}_D \right) - log \left( \widehat{\pi}_I \right) > log \left( \frac{\widehat{\pi}_R}{\widehat{\pi}_I} \right) $$
$$\Leftrightarrow log \left( \frac{\widehat{\pi}_D}{\widehat{\pi}_I} \right) > log \left( \frac{\widehat{\pi}_R}{\widehat{\pi}_I} \right) \Leftrightarrow 3.3 - 0.2 x > 1.0 + 0.3 x \Leftrightarrow 2.3 > 0.5x \Leftrightarrow 4.6 > x$$
Por lo tanto, el intervalo para $x$ cuando $ \widehat{\pi}_D > \widehat{\pi}_R$ es $x \in{(-\infty,4.6)}$

De otra manera, este intervalo significa el ingreso anual en dolares de una persona con una mayor probabilidad de preferencia por un presidente demócrata que por un republicano

\end{proof}

\item[(iii)]
\begin{proof}[\textcolor{blue}{Solución:}] Si se toma $\log \left(\frac{ \widehat{\pi}_D}{\widehat{\pi}_I}\right) = 3.3 - 0.2 x$ de la hipótesis. Esto lleva a que:
$$\frac{\widehat{\pi}_D}{\widehat{\pi}_I} = exp( 3.3 - 0.2 x ) \Rightarrow \widehat{\pi}_D = \widehat{\pi}_I exp( 3.3 - 0.2 x ) \Rightarrow \widehat{\pi}_I = \widehat{\pi}_D exp( 0.2 x - 3.3 )$$

Por lo tanto, una expresión para $\widehat{\pi}_I$ es $\widehat{\pi}_I = \widehat{\pi}_D exp( 0.2 x - 3.3)$
\end{proof}

\end{compactenum}$$\\$$

\noindent \textbf{2. } Encuentre la devianza para el modelo logístico multinomial.\\
\begin{proof}[\textcolor{blue}{Solución:}] Observe que $f(y_i,\pi_i) = P[ Y_{i1} = y_{i1}, \dots , Y_{iJ} = y_{iJ} ] = \pi_{i1}^{y_{i1}} \cdots \pi_{iJ}^{iJ}$. Entonces:

$$f( \overline{y}, \overline{\pi} ) = \prod_{i=1}^{n} f( y_i, \pi_i) = \prod_{i=1}^{n} \prod_{k=1}^{J} \pi_{ik}^{y_{ik}} \Rightarrow l( y, \pi) = log\left( \prod_{i=1}^{n} \prod_{k=1}^{J} \pi_{ik}^{y_{ik}} \right) = \sum_{i=1}^{n} log\left( \prod_{k=1}^{J} \pi_{ik}^{y_{ik}} \right) = \sum_{i=1}^{n} \sum_{k=1}^{J} log\left( \pi_{ik}^{y_{ik}} \right)$$
$$\Rightarrow l( y, \pi)= \sum_{i=1}^{n} \sum_{k=1}^{J} y_{ik} log\left( \pi_{ik} \right)$$
$$\Rightarrow D = 2 [ l( y, \pi ) - l( y, y ) ] = 2 [ \sum_{i=1}^{n} \sum_{k=1}^{J} y_{ik} log\left( \pi_{ik} \right) - \sum_{i=1}^{n} \sum_{k=1}^{J} y_{ik} log\left( y_{ik} \right) ] = \sum_{i=1}^{n} \sum_{k=1}^{J} y_{ik} log\left( \frac{\pi_{ik}}{y_{ik}}\right)$$

$\therefore D = \sum_{i=1}^{n} \sum_{k=1}^{J} y_{ik} log\left( \frac{\pi_{ik}}{y_{ik}}\right)$ 

\end{proof}

\noindent \textbf{3. }  Un modelo logístico acumulativo se ajustó a los datos  de una encuesta social, donde se considera la ideología política (poco liberal a muy liberal) como respuesta, y variable explicativa la religión profesada (protestante, católicos, judíos y otros). Considerando variables dummys para las primeras tres religiones, el ajuste del modelo fue: $\hat \alpha_1=- 1.03$, $\hat\alpha_2=- 0.13$, $\hat\alpha_3= 1.57$, $\hat\alpha_4 = 2.41$, $\hat \beta_1 = -1.27$, $\hat \beta_2= -1.22$ y $\hat \beta_3= -0.44$. Conteste lo siguiente\\

\begin{compactenum}
\item[(i)] ¿Cuántas categorías tiene la respuesta $Y$?. ¿Cuál grupo religioso se estima si se es a) el más liberal y b ) el más conservador?.
\item[(ii)]Utilizando cociente de momios compare la ideología política para los grupos religiosos  protestantes y católicos. Interprete.\\
\end{compactenum}

\begin{compactenum}
\item[(i)]
\begin{proof}[\textcolor{blue}{Solución:}] $Y$ resulta tener 4 categorías derivadas de la región que profesan. De esta manera, $P[Y_i \leq j] = e^{\alpha_j + x_1 \beta_1 + x_2 \beta_2 + x_3 \beta_3}$

Por lo cual, si se quiere saber el grupo religioso el cuál es más liberal, y el más conservador. Entonces, evaluando en cada categoría:

$$P[Y_i = 4] = P[Y_i \leq 4] - P[Y_i \leq 3] = e^{\alpha_4} - e^{\alpha3 - \beta_3} = e^{2.41} - e^{1.57+(-0.44)} = e^{2.41} - e^{1.13} \approx 8.0383$$
$$P[Y_i = 3] = P[Y_i \leq 3] - P[Y_i \leq 2] = e^{\alpha_3 + \beta_3} - e^{\alpha_2 + \beta_2} = e^{1.57 + (-0.44)} - e^{-0.13 + (-1.22)} = e^{1.13} - e^{-1.35} \approx 2.8364$$
$$P[Y_i = 2] = P[Y_i \leq 2] - P[Y_i \leq 1] = e^{\alpha_2 + \beta_2} - e^{\alpha_1 + \beta_1} = e^{-0.13 + (-1.22)} - e^{-1.03 + (-1.27)} = e^{-1.35} - e^{-2.30} \approx 0.1589$$
$$P[Y_i = 1] = P[Y_i \leq 1] = e^{-2.30} \approx 0.1002$$

De lo cual, se puede notar que una persona más liberal se estima que profesa otra religión no señalada, mientras que los menos liberales serían las personas que profesan el Protestantismo.

\end{proof}

\item[(ii)]
\begin{proof}[\textcolor{blue}{Solución:}] Para encontrar el cociente de momios se necesita tener $\pi_C$ y $\pi_P$, de lo cual se obtiene los siguiente:

$$\pi_C = P[Y_i = 2] = P[Y_i \leq 2] - P[Y_i \leq 1] = e^{\alpha_2 + \beta_2} - e^{\alpha_1 + \beta_1} = e^{-0.13 - 1.22} - e^{-1.03 - 1.27}= e^{-1.35} - e^{-2.30} \approx 0.1589$$

De igual manera se obtiene que:

$$\pi_P = P[Y_i = 1] = P[Y_i \leq 1] - P[Y_i \leq 0] = e^{-2.30} = 0.1002 \Rightarrow \pi_P \approx 0.1002$$

Entonces, el cociente de momios entre protestantes y católicos es:

$$\frac{\pi_P/ (1 - \pi_P)}{\pi_C / (1 - \pi_C)} = \frac{0.1002 / (1-0.1002)}{0.1590 / (1 - 0.1590)} = \frac{0.1002 / (0.8998)}{0.1590 / (0.8410)} \approx \frac{0.1113}{0.1890} \approx 0.5888$$

Esto nos quiere decir que si se escoge una persona que es poco liberal hay hasta un $58\%$ más de probabilidad de que sea protestante a que sea católico.  

\end{proof}

\end{compactenum}

\noindent \textbf{4. } (\texttt{R}) 	La tabla presenta los resultados que tuvieron pasajeros en vehículos accidentados, de acuerdo con el género, el lugar del accidente y el uso de cinturón de seguridad.  Las categorías de respuesta son: (1) no se lastimó, (2) lastimado, pero sin necesidad de ser transportado en ambulancia, (3) lastimado, transportado por ambulancia pero no hospitalizado, (4) lastimado, hospitalizado, pero sobrevivió y (5) muerto.  Ajusta un modelo considerando la variable respuesta no ordenada y un modelo considerándola ordenada.  Comenta tus resultados.

\begin{table}[htpb]
	\centering 
	\begin{tabular}{ccc|ccccc}
		& & & \multicolumn{5}{c}{respuesta}\\
		\hline
género	 &localización&	cinturón&	1&	2	&3&	4&	5\\
\hline
femenino &	urbana&	no&	7287&	175&	720&	91	&10\\
&&sí&	11587	&126&	577&	48&	8\\
&rural	&no&	3246&	73	&710&	159&	31\\
&&sí	&6134&	94&	564&	82&	17\\
masculino &	urbana	&no&	10381&	136&	566&	96&	14\\
&&sí&	10969&	83&	259&	37	&1\\
&rural	&no&	6123	&141&	710	&188&	45\\
& &sí&	6693&	74	&353&	74&	12\\
\hline
\end{tabular}
\end{table}


\end{document}


