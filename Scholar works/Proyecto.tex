\documentclass[12pt,a4paper,oneside]{article}
\usepackage[utf8]{inputenc}
\usepackage{amsmath}
\usepackage{amsthm}
\usepackage{amsfonts}
\usepackage{amssymb}
\usepackage{paralist}
\usepackage{graphicx}
\usepackage{hyperref}
\usepackage{booktabs}
\usepackage[top=2.5cm,left=2.5cm,right=2.5cm,bottom=2cm]{geometry}
\usepackage[usenames]{color}
\usepackage{subfigure}
\usepackage{hyperref}
\usepackage{float}

\newcommand{\var}{\text{var}}
\newcommand{\bb}{\boldsymbol}
\renewcommand{\baselinestretch}{1.1}


\usepackage{txfonts}
\begin{document}

\vspace{8cm}
\begin{center}

\line(1,0){1}

\vspace{3cm}
\begin{huge}
\textbf{Proyecto}: Calculadora de riesgos COVID-19\\
\end{huge}
\vspace{0.75cm}

\begin{large}
\textbf{Alumno}: Cristobal Bautista Henández, cristobal@ciancias.unam.mx\\
\end{large}
\vspace{0.4cm}

\begin{large}
\textbf{Profesor}: Graciela Martínez Sánchez, graciela.mtz@ciencias.unam.mx\\

\vspace{0.1cm}
\textbf{Ayudante}: José de Jesús Ojeda Gónzalez, jesusojeda@ciencias.unam.mx\\

\vspace{0.2cm}
\textbf{Materia}: Seminario de Estadística (Modelos Lineales Generalizados)\\

\vspace{1.5cm}

\begin{figure}[h]
\centering
\subfigure{\includegraphics[width=4cm]{UNAM.jpg}}
\hspace{4cm}
\subfigure{\includegraphics[width=4cm]{ciencias.png}}
\end{figure}

\end{large}
\end{center}

\newpage

\section{Introducción.}
El presente proyecto tiene el objetivo de determinar la probabilidad de que una persona sea intubada a partir de sus condiciones. Así mismo, se quiere determinar la cantidad de días que en que un paciente muere a partir de una condición inicial, contando desde la fecha de inicio de los síntomas hasta su defunción.

\section{Explicación de las variables.}
Los datos de dicho proyecto son provenientes de la base de datos pública del gobierno con respecto al COVID-19, que se encuentran en el siguiente link:

\begin{footnotesize}
\url{https://datos.gob.mx/busca/dataset/informacion-referente-a-casos-covid-19-en-mexico}
\end{footnotesize}

En el cual se encuentran 36 datos que recopilan de los pacientes que se les realizo o realizara una prueba al virus de COVID-19, todas estás variables están contenidas en tabla denominada variables.


\section{Análisis exploratorio}
Inicialmente se realiza un análisis para observar que variables podrían estar relacionadas con las variables dependientes. De lo cual para el primer caso, en el que nos interesa modelar la probabilidad de ser intubado. En la primera pagina se encuentra en el siguiente grupo de gráficos, es un análisis general de como lucen los datos en relación con las persona intubadas. De la cual, se puede notar una ligera relación, en especial con las variables Edad, Fecha (de inicio de los) síntomas, Neumonia, Diabetes y Sexo.

Por otro lado, para el segundo modelo en el cual se espera obtener la cantidad de días que una persona tarda en morir a partir de una variable se contrastan con las mismas variables del modelo anterior y corresponde a la segunda pagina de gráficos, en está se encuentran variables como el Sector, Renal crónica o diabetes

\includegraphics[width = 15cm]{Analisis1.png}

\includegraphics[width = 15cm]{Analisis2.png}

\section{Selección de modelo}
Para la selección del modelo que predice si una persona está intubada a partir de variables reportadas por el sistema de salud. Lo que se usa es un modelo logístico puesto que lo que nos interesa es si la persona necesita ser intubada o no, Para ello, se toman los casos de personas intubadas o aquellas que no lo son pero son positivos.

Se hicieron modelos contemplando todas las variables, las cuales al probarlas con el modelo con función ligando $logit$, y se seleccionaron aquellas variables que se podían asumir que no son cero, para ello, primero se utilizo el método \textit{backward} para minimizar el número de variables que explican la variables intubado y seguir siendo un modelo apropiado. Por tanto, las variables tienen relevancia para cambiar el modelo dependiendo del valor. Una vez seleccionada estas variables, se comparó el mismo modelo pero con las funciones ligando \textit{probit} y \textit{doble complementaria}, las cuales se fueron modelando de la misma manera y que finalmente se compararon con las estadística de Hosmer y Lemeshow, ya que nuestros datos no están ordenados

De las cuales, se puede ver que el mejor modelo que se ajusta a predecir la probabilidad de que una persona sea intubada es \textit{probit}. La cual, al tener un valor mayor valor en la prueba de rechazo para $\chi^2$ y menor \textit{p-value}, resulta como mejor modelo, a comparación de los otros dos. Aunque el modelo \textit{logit} también resultaría un buen candidato.

\begin{table}[H]
\resizebox{\textwidth}{!}{%
\begin{tabular}{|l|l|l|l|}
\hline
Modelo                   & p-valor de la prueba \textit{Hasmer y Lemeshow} & Valor estadístico $\chi^2$ & Grados de libertad \\ \hline
Logit                    & $7.911e^{-05}$                            & 32.396             & 8              \\
Probit                   & $5.601e^{-05}$                            & 33.229 & 8              \\
Logaritmo complementario & 0.01743                            & 18.555             & 8          \\   \hline
\end{tabular}%
}
\end{table}

Por otra parte para el modelo que estima el número de días que transcurre de la fecha de inicio de los síntomas hasta la muerte de dicho paciente, se decidió estudiarlo a partir de la condición de si la persona presentaba una condición renal crónica o no era el caso.
En este modelo se decidió estudiarlo como regresión Poisson, puesto que se modela el número de días a transcurrir.

\section{Análisis de Resultados}
De acuerdo a los análisis realizados lo que se encuentre es que una manera de obtener si una persona sera intubada depende del siguiente modelo.
$$\widehat{\pi}_{intubado_i} = \Phi(- 2.04176 + \beta{1j} Sector_{ij} + \beta_{2j} Entidad\_UM_{ij} - 0.86787 \ Neumonia_i + 0.00705 \ Edad_i $$$$+ \beta_{5j} Embarazo_{ij} - 1.59720 \ Habla\_lengua\_ind_i - 0.070175 \ Diabetes_i - 1.49562 \ EPOC_i$$
$$ - 0.10715 \ Obesidad_i - 0.23497 \ Cronica\_renal_i - 1.77718 \ UCI_i)$$
 
\textit{Neumonia, Diabetes, Obesidad, $Cronica\_renal$, $Habla\_lengua\_ind$, EPOC y UCI} son 0 si no se presenta y 1 si se presenta.

Para las variables \textit{$Entidad\_UM$, Embarazo} y \textit{Sector} se presentan en una tabla en el anexo, donde se ven los coeficientes del modelo, y en el cual toma el valor $1$, cuando se tiene la condición $j$.

\
De esta manera, lo que nos dice es que una persona va incrementando su probabilidad de ser intubada en algún momento si presenta algunas de las enfermedades descritas en el modelo, sobre todo neumonia que contiene el coeficiente más grande. A su vez, se es más sensible si se es hombre y/o va incrementando su edad y cuando al parecer la cada vez se tardan más en atender personas, puesto que como avanza el tiempo, la probabilidad desciende ligeramente.

Poniendo un ejemplo, si el paciente accedió a un hospital del IMSS, en Guanajuato con un cuadro de neumonía a la edad de 60 años habla lengua indígena, presenta una enfermedad renal crónica y entro a la UCI, entonces su probabilidad de ser intubada cambia es:

$$\Phi(- 2.04176 + 2.66213 (1) - 0.10032 (1) + 0.00705 (60) + 0.24767 (1) - 0.070175 (1) - 1.49562  - 0.10715 (1))$$
$$= \Phi(- 2.04176 + 2.66213 - 0.10032 + 0.00705 (60) + 0.24767 - 0.070175 - 1.49562  - 0.10715) = 0.3148231$$

La cual quiere decir que un $31\%$ de probabilidad que esta persona sea intubada.
Comparando con una mujer embarazada de 35 que padece de todas las condiciones de una mala salud, que fue a un hospital de la SEDENA en la CDMX, ingreso a una UCI y habla una lengua indígena.
$$\Phi(- 2.04176 + 2.63896 (1) + 0.02022 (1) + 0.00705 (35)) = 0.8062527$$

Lo cual nos dice que hay una probabilidad del $80\%$ de que la persona sea intubada.

Por otro lado, para encontrar el promedio de días que en que una persona muere dado que sufre una condición renal crónica. Lo cual, mueve a analizar el segundo modelo del segundo objetivo

Se tiene el siguiente modelo Poisson:

$\mu_i = exp\left[2.17383 + 0.20897 \ Cronica\_renal_j \right] = 8.791629$ \hspace{2.5cm} para $j=1,2$

$\mu_i = exp\left[2.17383 \right] = 8.791629$ \hspace{3.8cm} para $j=1$ o $Cronico\_renal=si$

$\mu_i = exp\left[2.17383 + 0.20897 \right] = 10.83487$ \hspace{2cm} para $j=2$ o $Cronico\_renal=no$

Con lo que se puede visualizar que hay un promedio de 9 días entre la fecha en que se empieza a mostrar síntomas y en la que la persona fallece. Por otra parte, en el caso de presentar una enfermedad crónica renal, este número de días incrementa hasta en un $23\%$

\section{Conclusiones}
Como se puede ver, de acuerdo a lo que se ha informado a lo largo de está epidemia, estar afectados con enfermedades crónicas se vuelve un factor importante en la salud del mismo, puesto que, tiene mayor probabilidad de ser intubado o de morir de manera más rápida.

A su vez, se muestra que en general los servicios de salud tienen la capacidad para intubar a los pacientes que sean necesarios, a excepción de los universitarios. Mientras que la edad afecta ligeramente mientras va incrementando.

\section{Material}
Se usaron las bases de datos proveniente del url ya citado
Todo esto se analizo en el programa de R y se utilizaron las paqueterías \textit{MASS} y \textit{ResourceSelection}.

\section{Anexo}

\begin{table}[h]
\renewcommand{\tablename}{Variables}
\caption{Las variables que se presentan en la base de datos. Para los casos \textit{NE, SN y si}  son \textit{No especificado, No Aplica} y \textit{Sin información}}
\resizebox{\textwidth}{!}{%
\begin{tabular}{|l|l|l|}
\hline
NOMBRE DE VARIABLE   & DESCRIPCIÓN DE VARIABLE                                                                                                                                                                                                                                                                                                                                                                                                                                                                                                                              & FORMATO O FUENTE                                                                                                       \\ \hline
FECHA\_ACTUALIZACION & La base de datos se alimenta diariamente, esta variable permite identificar la fecha de la ultima actualización.                                                                                                                                                                                                                                                                                                                                                                                                                                     & AAAA-MM-DD                                                                                                             \\ \hline
ID\_REGISTRO         & Número identificador del caso                                                                                                                                                                                                                                                                                                                                                                                                                                                                                                                        & TEXTO                                                                                                                  \\ \hline
ORIGEN               & \begin{tabular}[c]{@{}l@{}}La vigilancia centinela se realiza a través del sistema de unidades de salud monitoras de enfermedades \\ respiratorias (USMER). Las USMER incluyen unidades médicas del primer, segundo o tercer nivel de\\  atención y también participan como USMER las unidades de tercer nivel que por sus características\\  contribuyen a ampliar el panorama de información epidemiológica, entre ellas las que cuenten con \\ especialidad de neumología, infectología o pediatría. (Categorías en Catalógo Anexo).\end{tabular} & \begin{tabular}[c]{@{}l@{}}USMER = 0\\ No USMER = 1\end{tabular}                                                       \\ \hline
SECTOR               & Identifica el tipo de institución del Sistema Nacional de Salud que brindó la atención.                                                                                                                                                                                                                                                                                                                                                                                                                                                              & \begin{tabular}[c]{@{}l@{}}IMSS\\ IMSS-Bienestar\\ Universitario\\ SEDENA\\ ISSTE\\ etc.\end{tabular}                            \\ \hline
ENTIDAD\_UM          & Identifica la entidad donde se ubica la unidad medica que brindó la atención.                                                                                                                                                                                                                                                                                                                                                                                                                                                                        & \begin{tabular}[c]{@{}l@{}}TEXTO (Estados de la\\ Repúclica Mexicana)\end{tabular}                                     \\ \hline
SEXO                 & Identifica al sexo del paciente.                                                                                                                                                                                                                                                                                                                                                                                                                                                                                                                     & \begin{tabular}[c]{@{}l@{}}Mujer = 0\\ Hombre = 1\end{tabular}                                                         \\ \hline
ENTIDAD\_NAC         & Identifica la entidad de nacimiento del paciente.                                                                                                                                                                                                                                                                                                                                                                                                                                                                                                    & \begin{tabular}[c]{@{}l@{}}TEXTO (Estados de la\\ República Mexicana)\end{tabular}                                     \\ \hline
ENTIDAD\_RES         & Identifica la entidad de residencia del paciente.                                                                                                                                                                                                                                                                                                                                                                                                                                                                                                    & \begin{tabular}[c]{@{}l@{}}TEXTO (Estados de la\\ República Mexicana)\end{tabular}                                     \\ \hline
MUNICIPIO\_RES       & Identifica el municipio de residencia del paciente.                                                                                                                                                                                                                                                                                                                                                                                                                                                                                                  & \begin{tabular}[c]{@{}l@{}}TEXTO (Municipios de\\ la República Mexicana)\end{tabular}                                  \\ \hline
TIPO\_PACIENTE       & \begin{tabular}[c]{@{}l@{}}Identifica el tipo de atención que recibió el paciente en la unidad. Se denomina como ambulatorio si \\ regresó a su casa o se denomina como hospitalizado si fue ingresado a hospitalización.\end{tabular}                                                                                                                                                                                                                                                                                                               & \begin{tabular}[c]{@{}l@{}}Ambulatorio = 0\\ Hospitalizado = 1\end{tabular}                                            \\ \hline
FECHA\_INGRESO       & Identifica la fecha de ingreso del paciente a la unidad de atención.                                                                                                                                                                                                                                                                                                                                                                                                                                                                                 & AAAA-MM-DD                                                                                                             \\ \hline
FECHA\_SINTOMAS      & Idenitifica la fecha en que inició la sintomatología del paciente.                                                                                                                                                                                                                                                                                                                                                                                                                                                                                   & AAAA-MM-DD                                                                                                             \\ \hline
FECHA\_DEF           & Identifica la fecha en que el paciente falleció.                                                                                                                                                                                                                                                                                                                                                                                                                                                                                                     & AAAA-MM-DD                                                                                                             \\ \hline
INTUBADO             & Identifica si el paciente requirió de intubación.                                                                                                                                                                                                                                                                                                                                                                                                                                                                                                    & \begin{tabular}[c]{@{}l@{}}SI = 0\\ NO = 1\end{tabular} \\ \hline
NEUMONIA             & Identifica si al paciente se le diagnosticó con neumonía.                                                                                                                                                                                                                                                                                                                                                                                                                                                                                            & \begin{tabular}[c]{@{}l@{}}SI = 0\\ NO = 1\end{tabular} \\ \hline
EDAD                 & Identifica la edad del paciente.                                                                                                                                                                                                                                                                                                                                                                                                                                                                                                                     & NÚMERICA EN AŃOS                                                                                                       \\ \hline
NACIONALIDAD         & Identifica si el paciente es mexicano o extranjero.                                                                                                                                                                                                                                                                                                                                                                                                                                                                                                  & NACIONALIDAD                                                                                                           \\ \hline
EMBARAZO             & Identifica si la paciente está embarazada.                                                                                                                                                                                                                                                                                                                                                                                                                                                                                                           & \begin{tabular}[c]{@{}l@{}}SI = 0\\ NO = 1\end{tabular}                                                                                                      \\ \hline
HABLA\_LENGUA\_INDIG & Identifica si el paciente habla lengua índigena.                                                                                                                                                                                                                                                                                                                                                                                                                                                                                                     & \begin{tabular}[c]{@{}l@{}}SI = 0\\ NO = 1\end{tabular}                                                                                                      \\ \hline
DIABETES             & Identifica si el paciente tiene un diagnóstico de diabetes.                                                                                                                                                                                                                                                                                                                                                                                                                                                                                          & \begin{tabular}[c]{@{}l@{}}SI = 0\\ NO = 1\end{tabular}                                                                                                      \\ \hline
EPOC                 & Identifica si el paciente tiene un diagnóstico de EPOC.                                                                                                                                                                                                                                                                                                                                                                                                                                                                                              & \begin{tabular}[c]{@{}l@{}}SI = 0\\ NO = 1\end{tabular}                                                                                                      \\ \hline
ASMA                 & Identifica si el paciente tiene un diagnóstico de asma.                                                                                                                                                                                                                                                                                                                                                                                                                                                                                              & \begin{tabular}[c]{@{}l@{}}SI = 0\\ NO = 1\end{tabular}                                                                                                      \\ \hline
INMUSUPR             & Identifica si el paciente presenta inmunosupresión.                                                                                                                                                                                                                                                                                                                                                                                                                                                                                                  & \begin{tabular}[c]{@{}l@{}}SI = 0\\ NO = 1\end{tabular}                                                                                                      \\ \hline
HIPERTENSION         & Identifica si el paciente tiene un diagnóstico de hipertensión.                                                                                                                                                                                                                                                                                                                                                                                                                                                                                      & \begin{tabular}[c]{@{}l@{}}SI = 0\\ NO = 1\end{tabular}                                                                                                      \\ \hline
OTRAS\_COM           & Identifica si el paciente tiene diagnóstico de otras enfermedades.                                                                                                                                                                                                                                                                                                                                                                                                                                                                                   & \begin{tabular}[c]{@{}l@{}}SI = 0\\ NO = 1\end{tabular}                                                                                                      \\ \hline
CARDIOVASCULAR       & Identifica si el paciente tiene un diagnóstico de enfermedades cardiovasculares.                                                                                                                                                                                                                                                                                                                                                                                                                                                                     & \begin{tabular}[c]{@{}l@{}}SI = 0\\ NO = 1\end{tabular}                                                                                                      \\ \hline
OBESIDAD             & Identifica si el paciente tiene diagnóstico de obesidad.                                                                                                                                                                                                                                                                                                                                                                                                                                                                                             & \begin{tabular}[c]{@{}l@{}}SI = 0\\ NO = 1\end{tabular}                                                                                                      \\ \hline
RENAL\_CRONICA       & Identifica si el paciente tiene diagnóstico de insuficiencia renal crónica.                                                                                                                                                                                                                                                                                                                                                                                                                                                                          & \begin{tabular}[c]{@{}l@{}}SI = 0\\ NO = 1\end{tabular}                                                                                                      \\ \hline
TABAQUISMO           & Identifica si el paciente tiene hábito de tabaquismo.                                                                                                                                                                                                                                                                                                                                                                                                                                                                                                & \begin{tabular}[c]{@{}l@{}}SI = 0\\ NO = 1\end{tabular}                                                                                                      \\ \hline
OTRO\_CASO           & Identifica si el paciente tuvo contacto con algún otro caso diagnósticado con SARS CoV-2                                                                                                                                                                                                                                                                                                                                                                                                                                                             & \begin{tabular}[c]{@{}l@{}}SI = 0\\ NO = 1\end{tabular}                                                                                                      \\ \hline
RESULTADO            & \begin{tabular}[c]{@{}l@{}}Identifica el resultado del análisis de la muestra reportado por el  laboratorio de la Red Nacional de \\ Laboratorios de Vigilancia Epidemiológica (INDRE, LESP y LAVE). (Catálogo de resultados \\ diagnósticos anexo).\end{tabular}                                                                                                                                                                                                                                                                                    & \begin{tabular}[c]{@{}l@{}}+ = Positivo\\ - = Negativo\\ ? = Sospechoso\end{tabular}                                   \\ \hline
MIGRANTE             & Identifica si el paciente es una persona migrante.                                                                                                                                                                                                                                                                                                                                                                                                                                                                                                   & \begin{tabular}[c]{@{}l@{}}SI = 0\\ NO = 1\end{tabular}                                                                                                      \\ \hline
PAIS\_NACIONALIDAD   & Identifica la nacionalidad del paciente.                                                                                                                                                                                                                                                                                                                                                                                                                                                                                                             & TEXTO                                                                                                                  \\ \hline
PAIS\_ORIGEN         & Identifica el país del que partió el paciente rumbo a México.                                                                                                                                                                                                                                                                                                                                                                                                                                                                                        & TEXTO                                                                                                                  \\ \hline
UCI                  & Identifica si el paciente requirió ingresar a una Unidad de Cuidados Intensivos.                                                                                                                                                                                                                                                                                                                                                                                                                                                                     & \begin{tabular}[c]{@{}l@{}}SI = 0\\ NO = 1\end{tabular}                                                                                                      
\\ \hline
\end{tabular}%
}
\end{table}

\begin{table}[]
\renewcommand{\tablename}{Coeficientes}
\caption{}
\resizebox{\textwidth}{!}{%
\begin{tabular}{|l|l|l|l|l|}
\hline
                       & Estimado            & Error Estándar      & Estadístico          & p-value              \\ \hline
(Intercept)            & -2.04176781962493   & 57.9383016313097    & -0.0352403809248279  & 0.971888062893245    \\ \hline
SECTOREstatal          & 3.35408285845027    & 57.9360178333538    & 0.0578928787977437   & 0.953833955506345    \\ \hline
SECTORIMSS             & 2.66213983712601    & 57.9373976848773    & 0.0459485572963675   & 0.963351251895489    \\ \hline
SECTORIMSSB            & 2.79623956758372    & 57.9359761133944    & 0.0482643040674902   & 0.961505602610921    \\ \hline
SECTORISSTE            & 1.93182835035793    & 57.9371285033709    & 0.0333435294475379   & 0.973400641563462    \\ \hline
SECTORMuniciapal       & 2.56550618646229    & 57.9360369793688    & 0.0442816996160072   & 0.964679858955133    \\ \hline
SECTORPEMEX            & 2.46414678883821    & 57.9360105853215    & 0.0425322137983474   & 0.966074432079495    \\ \hline
SECTORPrivada          & 2.54104248035815    & 57.9366489221572    & 0.0438589826583214   & 0.965016810914624    \\ \hline
SECTORSEDENA           & 2.63896116187859    & 57.936080714622     & 0.0455495285377936   & 0.963669297773926    \\ \hline
SECTORSEMAR            & 2.90561305322399    & 57.9359550416757    & 0.0501521559648729   & 0.960001137548556    \\ \hline
SECTORSSA              & 4.08747834142811    & 57.9365247211399    & 0.070550975590993    & 0.943755128938688    \\ \hline
SECTORUniversitario    & -0.184087656441433  & 81.9337449711897    & -0.00224678679713913 & 0.9982073250114      \\ \hline
ENTIDAD\_UMBC          & -0.256877871661681  & 0.364723198581978   & -0.704309110745921   & 0.481240293765192    \\ \hline
ENTIDAD\_UMBS          & -0.0618013939709723 & 0.48964888234641    & -0.126215735804028   & 0.899561155607611    \\ \hline
ENTIDAD\_UMCC          & -1.07489915082474   & 0.474638556255564   & -2.26466884465656    & 0.0235330087256736   \\ \hline
ENTIDAD\_UMCL          & -0.355613956471047  & 0.503792041837034   & -0.705874501658129   & 0.480266185722412    \\ \hline
ENTIDAD\_UMCM          & -0.0598391462788802 & 0.58533142074823    & -0.102231221762173   & 0.918573146192687    \\ \hline
ENTIDAD\_UMCS          & -0.733707608942906  & 0.393408897376452   & -1.86500003898189    & 0.0621813987446246   \\ \hline
ENTIDAD\_UMCH          & -0.907099653464191  & 0.377337820469062   & -2.40394575962884    & 0.0162191795883564   \\ \hline
ENTIDAD\_UMDF          & 0.0202296826122888  & 0.358045116336914   & 0.056500372967671    & 0.954943198416174    \\ \hline
ENTIDAD\_UMDG          & 0.158730193346174   & 0.553626530211112   & 0.286709875131248    & 0.774334475643244    \\ \hline
ENTIDAD\_UMGT          & -0.100327439196462  & 0.390699902759303   & -0.256789004778099   & 0.797341658448629    \\ \hline
ENTIDAD\_UMGR          & -0.627215612331501  & 0.378392730330359   & -1.6575783889503     & 0.0974025946569581   \\ \hline
ENTIDAD\_UMHG          & -0.100165589712046  & 0.367883735603272   & -0.272275123954012   & 0.785410482126675    \\ \hline
ENTIDAD\_UMJC          & -0.116078259710986  & 0.408356428190757   & -0.284257211831528   & 0.77621327415976     \\ \hline
ENTIDAD\_UMMC          & -0.159510574012335  & 0.359680287578352   & -0.443478776905692   & 0.657419466789493    \\ \hline
ENTIDAD\_UMMN          & -0.234579390968748  & 0.394594384389099   & -0.594482334896677   & 0.552189553983199    \\ \hline
ENTIDAD\_UMMS          & -0.0026994860404764 & 0.367636856708305   & -0.00734280579114587 & 0.994141341273004    \\ \hline
ENTIDAD\_UMNT          & -0.518892754716739  & 0.453627866808219   & -1.14387318920183    & 0.25267623138504     \\ \hline
ENTIDAD\_UMNL          & -0.663743881628938  & 0.428196511521891   & -1.55009175406373    & 0.121119494943094    \\ \hline
ENTIDAD\_UMOC          & 0.0462217535621078  & 0.379139786198892   & 0.121912168663461    & 0.902968578316846    \\ \hline
ENTIDAD\_UMPL          & -0.462178299727245  & 0.365226532688228   & -1.26545652728297    & 0.205707722730016    \\ \hline
ENTIDAD\_UMQT          & -0.010043096550624  & 0.388439200833017   & -0.0258550026080949  & 0.979372990752927    \\ \hline
ENTIDAD\_UMQR          & -0.482737464721087  & 0.372774823772825   & -1.29498408673456    & 0.195325719105188    \\ \hline
ENTIDAD\_UMSP          & -0.479216199663853  & 0.445688649311676   & -1.07522639493727    & 0.282273380806685    \\ \hline
ENTIDAD\_UMSL          & -0.341949202480414  & 0.367123802351595   & -0.931427492006985   & 0.351632475304281    \\ \hline
ENTIDAD\_UMSR          & -0.83638702150793   & 0.423407422266558   & -1.97537165747033    & 0.0482259867495965   \\ \hline
ENTIDAD\_UMTC          & -1.02322711977477   & 0.37352759247835    & -2.73936153681626    & 0.00615586385101062  \\ \hline
ENTIDAD\_UMTS          & -0.98832226049235   & 0.45211488178769    & -2.18599807328718    & 0.0288157462891343   \\ \hline
ENTIDAD\_UMTL          & -0.849459144022615  & 0.391332185813889   & -2.17068560884129    & 0.0299549457788721   \\ \hline
ENTIDAD\_UMVZ          & -0.645662470963373  & 0.365526777815146   & -1.76638897654141    & 0.0773306132101995   \\ \hline
ENTIDAD\_UMYN          & -0.236910002690235  & 0.390107956267729   & -0.607293439889866   & 0.543656191116246    \\ \hline
ENTIDAD\_UMZS          & -0.763517487287565  & 0.451087245452474   & -1.69261599609561    & 0.0905285842826254   \\ \hline
NEUMONIANO             & -0.867869552434172  & 0.0619656663126899  & -14.0056519049557    & 1.43952703712064e-44 \\ \hline
EDAD                   & 0.00705265290857143 & 0.00119421133564989 & 5.90569918240925     & 3.51154224993972e-09 \\ \hline
EMBARAZONO             & 0.12670732801135    & 0.339014606207479   & 0.373751825706898    & 0.708588963633029    \\ \hline
EMBARAZOSN             & 0.247672931228954   & 0.338117383573813   & 0.73250576060632     & 0.463859927958791    \\ \hline
HABLA\_LENGUA\_INDIGNO & -0.178841760668712  & 0.111971551363686   & -1.59720713422849    & 0.110219542969902    \\ \hline
DIABETESNO             & -0.0701747062816444 & 0.0370696757545904  & -1.89304882908113    & 0.0583513762114406   \\ \hline
EPOCNO                 & -0.135918736801796  & 0.0908776357254028  & -1.49562360108586    & 0.134751769336759    \\ \hline
OBESIDADNO             & -0.10715853709824   & 0.0385340803447098  & -2.78087698317033    & 0.00542122757489137  \\ \hline
RENAL\_CRONICANO       & -0.234977740818289  & 0.0811645710667507  & -2.89507771346491    & 0.00379064674686537  \\ \hline
UCINO                  & -1.77718304387105   & 0.0433268628046359  & -41.0180412065491    & 0                    \\ \hline
\end{tabular}%
}
\end{table}

\end{document}

